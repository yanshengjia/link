\chapter{绪论}

\section{研究动机}
如今 Web 上的内容每天都在以指数级的速度增长\cite{limaye2010annotating},这也使得 Web 近年来成为世界上最大的数据集散地之一。
据估计 Web 上有超过141亿张表格,其中1.54亿张表格包含关系型数据并且仅 Wikipedia 就是大约160万关系型表格的来源。
可见Web 表格,换句话说是 Web 上的 HTML 表格,是关系型数据的一个重要来源和信息抽取 (Information Extraction) 系统的一个重要输入。
与普通文本不同,单张关系型表格就包含一系列高质量的关系实体并且在表格的列头中包含与实体相关的元数据。
Web 上关系型表格中蕴含的巨大财富和价值使得表格的语义解释 (Semantic Interpretation),也就是将 Web 表格转换成机器能够理解的知识这一任务成为热门的研究领域。\par

另一方面,诸如维基百科等知识共享社区的蓬勃发展和信息抽取技术的进步已经促成了大规模机器可读知识库的自动化建设。目前世界上已经出现了上百个领域不同、规模不一的知识库并且它们的规模每天都在飞速增长。知识库中包含着整个世界的实体,实体的语义类别及其相互关系的丰富信息。这样典型的例子包括 YAGO\cite{suchanek2007yago},DBpedia\cite{auer2007dbpedia} 和 Freebase\cite{bollacker2008freebase}。在中文知识库中较有影响力的就是由上海交通大学和东南大学共同建立的 zhishi.me\cite{niu2011zhishi}。\par

Web 上虽然蕴含着许多有价值的数据,但更多的却是各种原始并且充斥着噪声的数据,其中有些甚至是错误的。
这些数据大都是以自然语言的形式存在,然而由于自然语言表达的多样性与歧义性,使得它们很难被计算机直接处理或者理解。
面对如此规模庞大而嘈杂的数据,信息过载的现象每天都在发生。
信息过载意味着在找到想要的有用的信息之前,需要处理大量的无用数据,计算机获取有效信息的效率常常受到限制。
为了减轻信息过载带来的负面影响以及处理自然语言的多样性问题和歧义性问题,语义网 (Semantic Web) 的概念应运而生,旨在对现有万维网上的文档进行元数据 (Meta Data) 标注,使计算机能够理解词语和概念以及它们之间的逻辑关系。
将 Web 数据与知识库链接起来是非常有利于标注 Web 上的大规模数据的,并且有助于实现语义网的愿景\cite{berners2001}。
许多为了解释说明 Web 表格内含的语义的研究工作\cite{hignette2009fuzzy}\cite{limaye2010annotating}\cite{mulwad2013semantic}\cite{munoz2014using}\cite{syed2010exploiting}\cite{venetis2011recovering}是将其内容标注成 RDF 三元组。
这种标注的关键一步就是实体链接 (Entity Linking),将出现在 Web 表格单元格中的命名实体指称链接到其在给定知识库中对应的实体。\par

实体链接技术的发展可以带动许多不同的应用的发展,比如知识库补全,自然语言问答系统和语义搜索系统。
随着社会的发展,新的知识被创造出来并以数据的形式表现在 Web 上。
因此,利用这些新知识扩充现有的知识库显得越发重要。
然而,为了将这些新知识插入到现有知识库中,会不可避免地需要一个系统,来将命名实体指称,也就是与已抽取出的知识相关联的指称,链接到知识库中的相应实体。
例如,自然语言问答系统依靠它们支持的知识库来回答用户的问题。
为了回答``苹果公司创始人史蒂夫·乔布斯的诞生日期''的问题,该系统应首先利用实体链接技术将查询语句中的``史蒂夫·乔布斯''映射到美国企业家,而不是美国传记电影,
然后从知识库中直接取回名为``史蒂夫·乔布斯''的出生日期。
除此之外,实体链接对数据集成很有帮助,可以将不同页面、文档和站点上的实体信息进行集成。
可见,Web 表格上的实体链接技术很有价值并且拥有广阔的应用前景。\par


\section{研究现状}

在本节中,我会回顾了有关Web表格上语义注释的一些相关工作,通常会解决三个任务:实体链接(EL),列类型推断以及同一行中的实体之间的关系提取。 
Cafarella等人 [6]报道说,有超过1.5亿个Web表嵌入了高质量的关系数据,许多研究人员意识到Web表是可用于许多应用程序的重要来源,如信息提取和结构化数据搜索。 
因此,出现了关于Web表的语义注释的各种工作。\par

ignette等[9]提出了一种聚合方法,用于在给定本体中使用词汇表来注释Web表单的内容。
它首先注释单元格,然后注释列,最后关联这些列。
同样,Syed等[19]还提出了一种管道方法,其首先推断列的类型,然后将单元格值链接到给定的KB中的实体,最后选择列之间的适当关系。
 Zhang [22]设计了一个名为TableMiner的工具来注释Web表。 
 TableMiner仅关注列类型推理和EL,并且不能从Web表中提取关系。之后,Zhang [21]也提出了一些改进TableMiner的策略。 
 Limaye等人[10]和Mulwad等人[11]描述了可以分别联合模拟Web表的EL,列类型推断和关系提取任务的两种方法。
 我们的方法和这些工作之间的主要区别是我们不使用任何特定的信息来完成EL的任务,例如Web表的列标题和标题,KB中的实体类型,网页中的语义标记等。\par

在没有EL步骤的情况下,在Web表格上的语义注释的具体方案中也存在一些工作。 
在Venetis等人的工作中 [20],他们的方法削弱了EL的影响,直接推断了列的类型,并且通过大规模的关系数据库和关系数据库来确定关系,它们都是由网页构建的,但通常不可用 大部分研究人员。 
此外,Mun oz et al。 [12]提出了一种从维基百科表中挖掘RDF三元组的方法。
 在这项工作中,他们可以通过内部链接和文章标题直接识别维基百科中的实体。\par

Shen等人最接近我们的做法 [16]和Bhagavatula等人[2]。
 Shen et al。 [16]尝试将列表类Web表(多行与一列)中的字符串提交到给定KB中的实体。
  Bhagavatula等 [2]提出了TabEL,一个表实体链接系统,它使用集体分类技术来集体消除给定Web表中的所有提及。 
  这两个工作都不使用EL的任何特定信息,并且可以应用于任何KB。 在这里,为了提高Web表格中的EL的质量,我们专注于具有多个链接KB而不是单个KB的EL。\par

\section{本文贡献}



